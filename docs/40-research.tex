\section{Исследовательская часть}
В данном разделе будет проведено нагрузочное тестирование разработанного сервера и сравнение его показателей производительности с NGINX.

\subsection{Описание измерений}
В качестве эталона для сравнения показателей производительности статического сервера был выбран NGINX [7], являющийся HTTP-сервером общего назначения. Конфигурация NGINX приведена в листинге \ref{code:nginx-conf}, docker-compose файл для развертывания nginx в листинге \ref{code:nginx-compose}

\begin{code}
	\captionof{listing}{Конфигурация NGINX.}
	\label{code:nginx-conf}
	\inputminted
	[
	frame=single,
	framerule=0.5pt,
	framesep=20pt,
	fontsize=\small,
	tabsize=4,
	linenos,
	numbersep=5pt,
	xleftmargin=10pt,
	breaklines,
	]
	{text}
	{code/nginx.conf}
\end{code}

\newpage

\begin{code}
	\captionof{listing}{docker-compose файл для развертывания nginx.}
	\label{code:nginx-compose}
	\inputminted
	[
	frame=single,
	framerule=0.5pt,
	framesep=20pt,
	fontsize=\small,
	tabsize=4,
	linenos,
	numbersep=5pt,
	xleftmargin=10pt,
	breaklines,
	]
	{text}
	{code/docker-compose.nginx.yml}
\end{code}

Ниже приведены технические характеристики устройства, на котором проводилось тестирование.
\begin{itemize}
	\item Операционная система: macos 15.
	\item Объём оперативной памяти: 32 Гб.
	\item Процессор: M1.
\end{itemize}

Тестирование проводилось на ноутбуке, включенном в сеть электропитания. Во время тестирования ноутбук был нагружен только встроенными приложениями окружения, а также непосредственно системой тестирования.
Для замеров метрик производительности сравниваемых веб-серверов использовалась утилита Apache Benchmark (ab)\cite{ab}.

\subsection{Результаты измерений}
Результаты нагрузочного тестирования разработанного сервера и NGINX представлены в таблицах \ref{tab:1}-\ref{tab:3}. Для оценки производительности измерялось количество обработанных запросов в секунду (Requests Per Second, RPS).

\begin{table}[h]
	\centering
	\caption{RPS при обработке 10000 запросов на файл размером 10 Кб}
	\label{tab:1}
	\begin{tabular}{|c|c|c|}
		\hline
		Число клиентов & Разработанный сервер & NGINX \\ \hline
		10 & 4927.22 & 3310.99 \\ \hline
		100 & 3386.79 & 3785.10 \\ \hline
		300 & 900.58 & 1001.57 \\ \hline
	\end{tabular}
\end{table}

\begin{table}[h]
	\centering
	\caption{RPS при обработке 1000 запросов на файл размером 1 Мб}
	\label{tab:2}
	\begin{tabular}{|c|c|c|}
		\hline
		Число клиентов & Разработанный сервер & NGINX \\ \hline
		10& 262.78 & 238.59 \\ \hline
		100 & 232.62 & 278.33 \\ \hline
		300 & 29.93 & 147.92 \\ \hline
	\end{tabular}
\end{table}

\begin{table}[h]
	\centering
	\caption{RPS при обработке 10 запросов на файл размером 104 Мб}
	\label{tab:3}
	\begin{tabular}{|c|c|c|}
		\hline
		Число клиентов & Разработанный сервер & NGINX \\ \hline
		1 & 1.65 & 2.99 \\ \hline
		5 & 3.01 & 3.03 \\ \hline
		10 & 3.00 & 3.04 \\ \hline
	\end{tabular}
\end{table}

\subsection{Выводы}
При обработке относительно лёгких запросов RPS сравниваемых серверов не отличался более чем на 10-15\%, в некоторых случаях, разработанный сервер даже превосходит NGINX, однако при увеличении числа конкурентных запросов производительность разработанный сервер проигрывает по эффективности серверу NGINX. При увеличении размера запрашиваемых файлов
производительность NGINX оказалась выше на 10-15\%, при условии 100 конкуретных запросов. При работе с относительно большими файлами(100мб) оба сервера показали примерно одинаковую производительность и пропускную способность.

\pagebreak