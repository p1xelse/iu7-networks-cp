\section*{ВВЕДЕНИЕ}
\addcontentsline{toc}{section}{ВВЕДЕНИЕ}

Термин <<веб-сервер>>  может относиться как к аппаратному, так и к
программному обеспечению.

С точки зрения аппаратного обеспечения веб-сервер -- это компьютер,
на котором хранятся программное обеспечение веб-сервера и файлы компонентов веб-сайта (например, HTML-документы, изображения, стили CSS и файлы
JavaScript). Веб-сервер подключается к Интернету и поддерживает физический
обмен данными с другими устройствами, подключенными к Интернету. 

С точки зрения программного обеспечения веб-сервер включает в себя
различные компоненты, которые контролируют доступ веб-пользователей к размещённым файлам. Одним из таких компонентов является HTTP-сервер.

HTTP-сервер -- это программное обеспечение, которое работает
с URL-адресами по протоколу HTTP, доставляя содержимое веб-сайтов на
устройства конечных пользователей. Доступ к HTTP-серверу можно получить
через доменные имена веб-сайтов, которые он хранит.

Статический сервер -- это сервер, который предоставляет клиентам
исключительно статические файлы. Это могут быть файлы HTML, CSS,
JavaScript, изображения, видео, аудио и любые другие файлы, которые клиенты
скачивают и используют в исходном виде, без какой-либо обработки на стороне
сервера. 

Статические серверы существенно отличаются от динамических серверов, которые могут выполнять серверный код, чтобы формировать вебстраницы «на лету» или предоставлять другие функции, такие как обработка
форм, взаимодействие с базами данных и выполнение других операций. 

Целью данной работы является разработка статического сервера. Для достижения поставленной цели необходимо решить следующие задачи.

\begin{enumerate}[leftmargin=1.6\parindent]
	\item Провести анализ предметной области и формализовать задачу.
	
	\item Спроектировать структуру программного обеспечения.
	
	\item Реализовать программное обеспечение, которое будет обслуживать контент,
	хранящийся во вторичной памяти.
	
	\item Провести нагрузочное тестирование и сравнение с распространёнными аналогами.
\end{enumerate}

\pagebreak
