\section{Технологическая часть}
В данном разделе будут описаны детали реализации и развёртывания ПО, а также приведены листинги кода.
\subsection{Средства реализации}
Для реализации статического сервера был выбран язык C в соответствии
с заданием на курсовую работу. Для мультиплексирования клиентских соединений был выбран мультиплексор epoll, для параллелизации обработки запросов -- пул потоков.
Для контроля качества кода использовался отладчик использования памяти valgrind \cite{valgrind}, для отладки работы с потоками использовалась утилита helgrind\cite{helgrind}, входящая в valgrind.


\subsection{Детали реализации}
\subsubsection{Сервер}
В листингах \ref{code:create} и \ref{code:loop} представлены функции, реализующие обработку клиентских запросов с использованием мультиплексора epoll.

\begin{code}
	\captionof{listing}{Функция создания серверного сокета.}
	\label{code:create}
	\inputminted
	[
	frame=single,
	framerule=0.5pt,
	framesep=20pt,
	fontsize=\small,
	tabsize=4,
	linenos,
	numbersep=5pt,
	xleftmargin=10pt,
	breaklines,
	firstline=1, 
	lastline=36, 
	]
	{text}
	{code/epoll.c}
\end{code}

\newpage

\begin{code}
	\captionof{listing}{Функция, содержащая основной цикл сервера с обработкой соединений.}
	\label{code:loop}
	\inputminted
	[
	frame=single,
	framerule=0.5pt,
	framesep=20pt,
	fontsize=\small,
	tabsize=4,
	linenos,
	numbersep=5pt,
	xleftmargin=10pt,
	breaklines,
	firstline=38, 
	]
	{text}
	{code/epoll.c}
\end{code}

\subsubsection{Пул потоков}
В листингах \ref{code:thpool1} - \ref{code:thpool3} представлены функции, реализующие пул потоков.

\begin{code}
	\captionof{listing}{Функция создания пула потоков.}
	\label{code:thpool1}
	\inputminted
	[
	frame=single,
	framerule=0.5pt,
	framesep=20pt,
	fontsize=\small,
	tabsize=4,
	linenos,
	numbersep=5pt,
	xleftmargin=10pt,
	breaklines,
	firstline=1, 
	lastline=52, 
	]
	{text}
	{code/thpool.c}
\end{code}

\begin{code}
	\captionof{listing}{Функция добавления задачи в очередь на выполнение.}
	\label{code:thpool2}
	\inputminted
	[
	frame=single,
	framerule=0.5pt,
	framesep=20pt,
	fontsize=\small,
	tabsize=4,
	linenos,
	numbersep=5pt,
	xleftmargin=10pt,
	breaklines,
	firstline=54, 
	lastline=102, 
	]
	{text}
	{code/thpool.c}
\end{code}

\begin{code}
	\captionof{listing}{Функция обработки очереди задач.}
	\label{code:thpool3}
	\inputminted
	[
	frame=single,
	framerule=0.5pt,
	framesep=20pt,
	fontsize=\small,
	tabsize=4,
	linenos,
	numbersep=5pt,
	xleftmargin=10pt,
	breaklines,
	firstline=149, 
	lastline=188, 
	]
	{text}
	{code/thpool.c}
\end{code}

\subsection{Поддерживаемые запросы}
Разработанный веб-сервер обрабатывает запросы GET и HEAD. В первом
случае клиент получает в теле ответа запрошенный файл, во втором — только
заголовки ответа Content-Type и Content-Length. Ниже перечислены поддерживаемые статусы ответов сервера.
\begin{itemize}[label*=---]
	\item 200 --- успешное завершение обработки запроса.
	\item 403 --- доступ к запрошенному файлу запрещён.
	\item 404 --- запрашиваемый файл не найден.
	\item 405 --- неподдерживаемый HTTP-метод (POST, PUT и т.д.).
	\item 500 --- внутренняя ошибка сервера.
\end{itemize}

Разработанный сервер поддерживает следующие форматы файлов (типы контента, Content-Type):

\begin{itemize}[label*=---]
	\item html (text/html);
	\item css (text/css);
	\item js (text/javascript);
	\item png (image/png);
	\item jpg (image/jpg);
	\item jpeg (image/jpe);
	\item gif (image/gif);
	\item svg (image/svg);
	\item swf (application/x-shockwave-flash);
\end{itemize}

\subsection{Cборка и развертывание}
Запуск разработанного приложения осуществлялся с помощью системы
контейнеризации Docker. Файл для сборки образа приложения представлен
в листинге \ref{code:dockerfile}, файл docker-compose для развертывания в листингке \ref{code:compose}.
\newpage
\begin{code}
	\captionof{listing}{Dockerfile образа приложения.}
	\label{code:dockerfile}
	\inputminted
	[
	frame=single,
	framerule=0.5pt,
	framesep=20pt,
	fontsize=\small,
	tabsize=4,
	linenos,
	numbersep=5pt,
	xleftmargin=10pt,
	breaklines,
	]
	{text}
	{code/Dockerfile}
\end{code}

\begin{code}
	\captionof{listing}{docker-compose файл для развертывания приложения.}
	\label{code:compose}
	\inputminted
	[
	frame=single,
	framerule=0.5pt,
	framesep=20pt,
	fontsize=\small,
	tabsize=4,
	linenos,
	numbersep=5pt,
	xleftmargin=10pt,
	breaklines,
	]
	{text}
	{code/docker-compose.yml}
\end{code}

\pagebreak